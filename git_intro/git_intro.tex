\documentclass{gittalk}
%\documentclass[handout,notes]{gittalk}

\title{Introduction to Git}
\author{Andreas Gerdes}
\date{}
\usepackage{tcolorbox}
\usepackage{upquote}

% Exercises should a have a different background color to
% highlight the hands-on nature of the slide:
\newenvironment{exerciseframe}
{
\setbeamercolor{background canvas}{bg=orange!20}
\begin{frame}
}
{
\end{frame}
}

% Commands could be highlighted by a different background color:
\newcommand{\hlcommand}[1]{ %
\colorbox{base3}{\small \texttt{#1}}
}

\begin{document}
%============================================================================
% Titlepage

\begin{frame}
  \titlepage
  %\begin{textblock}{10}(0,-5)
  %  \includegraphics[height=40mm]{./logo.pdf}
  %\end{textblock}  
\end{frame}
%============================================================================

\section*{Motivation}

\begin{frame}[fragile]
\frametitle{Motivation - Why use version control?}
Versions in file names: does this look familiar?
\begin{lstlisting}
$ ls
file    file.2         file.keep  file.old.2
file~   file.20090803  file.new   file.save
file.1  file.bak       file.old
\end{lstlisting}
This is better than nothing, however what happened between the different
versions?  Which file is actually the most current?\\[1em]
\textbf{Version control is a way to}
\begin{itemize}
  \item keep a backup of changing files
  \item store a history of those changes
  \item and manage merging of changes in versions\\
        with different change sets.
\end{itemize}
\end{frame}

\begin{frame}
\frametitle{Use of version control}
%{\large \alert{Who uses version control?}}
\textbf{Who uses version control?}\\[0.5em]
Everyone who would like to access an old version of a document.  Or everyone
to whom such things have happened:
\begin{itemize}
    \item It would be nice to have the version from 2 hours ago \ldots
    \item I wrote that really well three days ago.  How did that go
        again?
    \item Oh no!  I deleted the file!
\end{itemize}
\vspace*{1em}
%{\large \alert{Where is version control used?}}
\textbf{Where is version control used?}\\[0.5em]
\begin{itemize}
\item Software development
\item Text and document processing/writing
\item Graphic design
\item System administration
\end{itemize}
\end{frame}

\begin{frame}
\frametitle{Use of version control (cont.)}
%{\large \alert{What kinds of files should be kept under version control?}}
\textbf{What kinds of files should be kept under version control?}\\[0.5em]
\begin{itemize}
\item Any kind of file which will be changed
\item Mainly text files
    \begin{itemize}
    \item Program code, documentation
    \item Theses, dissertations
    \item Configuration files
    \end{itemize}
\item Binary files also possible
    \begin{itemize}
    \item Graphics files: \texttt{.png}, \texttt{.tiff}
    \item Documents: \texttt{.odt}, (\texttt{.pdf})
    \end{itemize}
\end{itemize}
\vspace*{1em}
%{\large \alert{What shouldn't be kept under version control?}}
\textbf{What shouldn't be kept under version control?}\\[0.5em]
\begin{itemize}
\item Automatically generated files, e.g.: \texttt{.o}, \texttt{.log},
    \texttt{.pdf}
\item Editor backup files, e.g.: \texttt{file\~}, \texttt{file.bak}
\end{itemize}
\end{frame}

\begin{frame}[fragile]
\frametitle{Course prerequisites}
\begin{tcolorbox}
Bash
\end{tcolorbox}
\vspace*{1em}
\begin{tcolorbox}[title=Text editor]
\textbf{Mac:} Text Wrangler or Text Mate 2\\
\textbf{Windows}: Notepad++\\
\textbf{Linux}: gedit, leafpad, kate, nano, vim \dots
\end{tcolorbox}
\vspace*{1em}
\begin{tcolorbox}
Git
\end{tcolorbox}
\end{frame}

%============================================================================
% Table of contents

\begin{frame}
  \frametitle{Agenda}
  \tableofcontents
\end{frame}

%============================================================================
\section{Git history and design}

\begin{frame}
\frametitle{Git history and concept}
\begin{itemize}
  \item Originally written for Linux kernel development.
  \item All Linux kernel developers used to be able to use the proprietary
        \emph{Bitkeeper} version control system for free.
  \item In 2005 there were further restrictions put on Bitkeeper so that
        it wasn't as free as it used to be.
  \item Linus Torvalds was uneasy with the situation and decided to write his
        own tool.
\end{itemize}
\vspace*{1em}
\begin{itemize}
  \item Basically a versioned file system
  \item The version control system is developed on top of this
  \item More than 130 commands
  \item \emph{porcelain} and \emph{plumbing} commands
  \item Distributed repository model can be used centrally
  \item Very fast!
\end{itemize}

\end{frame}

\begin{frame}
\frametitle{Git design}
\begin{itemize}
  \item Distributed development
  \item Scalable up to thousands of developers
  \item Fast and efficient
  \item Maintain integrity and privacy
  \item Enforced responsibility
  \item Immutable objects
  \item Atomic operations
  \item Support and promote development with branches
  \item Complete repositories
  \item Clean design
  \item Free as in freedom
\end{itemize}
\end{frame}

%============================================================================
\section{Warm up (configuration and getting help)}

\begin{frame}[fragile]
\frametitle{Get help}

\begin{tcolorbox}[title=git -{}-help]
Also possible: \hlcommand{man git}, \hlcommand{git help}
\end{tcolorbox}

\begin{lstlisting}[basicstyle=\tiny\ttfamily\color{black}]
$ git --help
usage: git [--version] [--exec-path[=<path>]] [--html-path] [--man-path] [--info-path]
           [-p|--paginate|--no-pager] [--no-replace-objects] [--bare]
           [--git-dir=<path>] [--work-tree=<path>] [--namespace=<name>]
           [-c name=value] [--help] <command> [<args>]

The most commonly used git commands are:
   add        Add file contents to the index
   bisect     Find by binary search the change that introduced a bug
   branch     List, create, or delete branches
   checkout   Checkout a branch or paths to the working tree
   clone      Clone a repository into a new directory
   commit     Record changes to the repository
   diff       Show changes between commits, commit and working tree, etc
   fetch      Download objects and refs from another repository
   grep       Print lines matching a pattern
   init       Create an empty git repository or reinitialize an existing one
   log        Show commit logs
   merge      Join two or more development histories together
   mv         Move or rename a file, a directory, or a symlink
   [...]

\end{lstlisting}
\end{frame}

\begin{frame}[fragile]
\frametitle{Configure Git}
\begin{tcolorbox}[title=git config]
Important for the history of a file:\\
Who made a change and when?
\end{tcolorbox}
\vspace*{2em}
\begin{lstlisting}
git config --global user.name "Jon Doe"
git config --global user.email "jon.doe@email.com"
git config --global core.editor "vim"
git config --global color.ui True
\end{lstlisting}
\end{frame}

%============================================================================
\section{Basic commands (init, status, add, commit)}

\begin{frame}[fragile]
\frametitle{Create a (local) repository}
\begin{tcolorbox}[title=git init]
Initialize an empty Git repository
\end{tcolorbox}
\vspace*{0.5em}
\begin{lstlisting}
$ mkdir myDir
$ cd myDir/
$ git init
Initialized empty Git repository in /home/dog/myDir/.git/
$ ls -la
total 36
drwxr-xr-x  3 dog dog  4096 Jan 30 02:27 .
drwx------ 57 dog dog 24576 Jan 30 02:27 ..
drwxr-xr-x  7 dog dog  4096 Jan 30 02:27 .git
$ git status
# On branch master
#
# Initial commit
#
nothing to commit (create/copy files and use "git add" to track)
$
\end{lstlisting}
\end{frame}


\begin{exerciseframe}
\frametitle{Exercise: help, config, init \hfill (10 minutes)}
\begin{itemize}
  \item Explore \hlcommand{git help}.
  \item Configure Git with your name and e-mail.
  \item Create a folder named 'git\_intro' in your \$HOME.
  \item Initialize a Git repository in 'git\_intro'.
  \item Explore the \hlcommand{.git} subdirectory.
  \item Tell me some interesting thing you find.
\end{itemize}
\vspace*{0.5em}
\pause
\begin{itemize}
  \item What happens when you rename 'git\_intro' to something else?
  \item What happens when you delete the \hlcommand{.git} subdirectory?
  \item What happens when you call \hlcommand{git init} a second time 
  in a Git directory?
  \item Type \hlcommand{git config -{}-list} in your \texttt{\$HOME} 
  directory!
  \item Type \hlcommand{git config -{}-list} in your Git directory!
\end{itemize}
\end{exerciseframe}

\begin{frame}
\frametitle{Good to know}
\begin{itemize}
  \item Git help works also for specific operations:
  \begin{itemize}
    \item \hlcommand{git help add} = \hlcommand{man git-add}
    \item \hlcommand{git help config} = \hlcommand{man git-config}
    \item \hlcommand{git help init} = \hlcommand{man git-init}
  \end{itemize}
  \qquad \dots
  \item You can either use \hlcommand{git config <parameter>}\\
        or edit the \hlcommand{.gitconfig} or \hlcommand{./.git/config} text files.\\
        \begin{itemize}
        \item The same is true for many other operations
        \item Not every configuration option has (yet)\\
        a command line tool to change it
        \item Git is still being developed.
        \end{itemize}
\end{itemize}
\end{frame}

%============================================================================

\begin{frame}
\frametitle{How Git views content}
% Picture from Katy
\begin{center}
    %\resizebox{\textwidth}{!}{
        \begin{tikzpicture}
[
gitstate/.style={rectangle, minimum width=2cm, fill=orange!50, draw=red, thick},
link/.style={->, shorten <=1pt, shorten >=1pt, >=stealth, semithick},
label/.style={black}
]

\fill [green!20, rounded corners] (-2.3,-3.5) rectangle (2.3,3.5);
\node[text=green!70] at (0,3) {\textbf{Staging area / Index}};
\node[text=green!50] at (0,-3) {\textbf{\small staged = indexed = cached}};
\fill [blue!20, rounded corners] (3,-3.5) rectangle (6,3.5);
\node[text=blue!70] at (4.5,3) {\textbf{Repository}};

% file states 
\node at (-4, 2) [gitstate] (untracked) { untracked };
\node at (-4, -2) [gitstate] (modified) { modified };
\node at (0, 0) [gitstate] (staged) { staged };
\node[left] at (-1,0) {\hlcommand{git add}};

\node at (4.5, 0) [gitstate] (committed) { committed };

% links between the states
\draw [link] (untracked.south east) to (staged.north west);
\draw [link] (modified.north east) to (staged.south west);
\draw [link] (staged) to node [label,auto,swap] {\hlcommand{git commit}} (committed);

\end{tikzpicture}

    %}
\end{center}
\end{frame}

\begin{frame}
\frametitle{State of a repository}
\begin{tcolorbox}[title=git status]
Shows if you have untracked, modified\\ or staged (= to be committed)
files.\\[0.5em]
In case of a cloned repository \hlcommand{git status} also shows if you are
ahead of your remote (compared to last sync).
\end{tcolorbox}
\vspace*{0.5em}
See also: \hlcommand{git help status}
\end{frame}

\begin{frame}
\frametitle{Add files to the next commit}
\begin{tcolorbox}[title=git add]
Adds files to the staging area.\\[0.5em]
If a file is added once, Git is aware of it and will show it as \emph{modified}
instead of \emph{untracked} if you have made a modification.\\[0.5em]
A modified file is not automatically staged. You have to call\\
\hlcommand{git add} every time you want to commit the changes.
\end{tcolorbox}
\vspace*{0.5em}
In Git \emph{add} has a different meaning compared to Subversion.\\[0.5em]
Here it means: Mark this file to be committed in my next snapshot.\\[0.5em]
In Subversion \emph{add} means: Add this file to version control.
\end{frame}

\begin{frame}
\frametitle{Make a snapshot}
\begin{tcolorbox}[title=git commit]
Put files from staging area into the repository.\\[0.5em]
This command does not care about the state\\
of the working directory!\\[0.5em]
You could delete all files and still commit them\\
as long as they were staged.
\end{tcolorbox}
\vspace*{0.5em}
In Git \emph{commit} is a local operation.\\[0.5em]
In Subversion \emph{commit} alias \emph{ci} always connects to the (central) server.
\end{frame}

\begin{exerciseframe}
\frametitle{Exercise: status, add, commit \hfill (10 minutes)}
\begin{itemize}
  \item Create two files in your Git repository:
  \begin{itemize}
    \item \texttt{echo 'Red is a nice color.' > red.txt}
    \item \texttt{echo 'Blue is a nice color.' > blue.txt}
  \end{itemize}
  \item Add \texttt{red.txt} to the staging area
  \item Make an initial commit that contains content of 
  \texttt{red.txt}.
  \begin{itemize}
    \item Now modify \texttt{red.txt} again (add a second line: 'It is 
    the color of a rose.')
    \item Call \hlcommand{git status}\\
          (one file should be modified, the other one still untracked)
    \item Now: Run \hlcommand{git commit -a} without staging anything 
    first!
    \item What happens? 
  \end{itemize}
  \item Add \texttt{blue.txt} to the staging area.
  \item Can you unstage a file that was added with \hlcommand{git add}?
  \item Stage and commit \texttt{blue.txt} using:\\
        \hlcommand{git add blue.txt}\\
        \hlcommand{git commit -m 'Adds blue.txt to the project.'}
\end{itemize}
\end{exerciseframe}


\begin{frame}
\frametitle{Recommendations for commits}
\begin{itemize}
  \item A commit should contain a single, self contained idea.
  \item The commit message should contain a short description of the
    idea or change being made in this commit.
    \begin{itemize}
	  \item A one line \emph{subject line}
	  \item An optional text describing more details of the change.
	  \item The \emph{Why} of the change is important.
	  \item Remember: this is a communication exercise.
    \end{itemize}
  \item Commits should be as small as possible (\emph{atomic
    commits}).
\end{itemize}
\pause
\vspace*{0.5em}
\textbf{After the git commit command}
\begin{itemize}
  \item The first line of the commit message is shown
  \item There are no revision numbers. Commits are
        specified via SHA1 hashes. The first part of the SHA1 hash is also
        shown. This can be used as a reference to the commit.
  \item What else is shown?
\end{itemize}
\end{frame}

\begin{frame}[fragile]
\frametitle{Good to know}
\textbf{Extensions to bash make life easier}
\vspace*{0.5em}
\begin{lstlisting}[basicstyle=\tiny\ttfamily]
export VISUAL=vim

GIT_PS1_SHOWDIRTYSTATE="true"
GIT_PS1_SHOWSTASHSTATE="true"
GIT_PS1_SHOWUNTRACKEDFILES="true"

if [ -e /etc/bash_completion.d/git ]
then
. /etc/bash_completion.d/git
export PS1='\[\033[01;32m\]\u@\h\[\033[00m\]:\[\033[01;34m\]\w\[\033[0;31m\]$(__git_ps1 "\[\033[0;31m\] (%s)\[\033[0;31m\]")\[\033[00m\]\$ '
else
export PS1='\[\033[01;32m\]\u@\h\[\033[00m\]:\[\033[01;34m\]\w\[\033[00m\]\$ '
fi
\end{lstlisting}
\end{frame}

%============================================================================
\section{Explore history and changes (log, diff, show)}

\begin{frame}[fragile]
\frametitle{Browse the history}
\begin{tcolorbox}[title=git log]
Shows previous commits (message, author, date)
\end{tcolorbox}
\vspace*{1em}
There is also a \hlcommand{tig} command that provides a nice ncurses interface
(has to be installed separately).\\[1em]
Summary view with
\begin{lstlisting}[basicstyle=\normalsize\ttfamily]
git log --oneline
\end{lstlisting}
\vspace*{1em}
See diffs for a specific file:
\begin{lstlisting}[basicstyle=\normalsize\ttfamily]
git log -p <file>
\end{lstlisting}
\end{frame}

\begin{frame}[fragile]
\frametitle{Compare versions}
\begin{tcolorbox}[title=git diff]
Compares different versions of (text) files.\\[0.5em]
Default:\\Version in current working directory vs. last committed version.\\
(other modes possible, see git help diff)
\end{tcolorbox}
\vspace*{1em}
Show the diff between staging area and last commit:
\begin{lstlisting}[basicstyle=\normalsize\ttfamily]
git diff --cached [<path>...]
\end{lstlisting}
\textbf{In other words:}\\
\hlcommand{git diff -{}-cached} \textbf{shows you what will be committed next.}\\[1em]
Show the diff between two arbitrary commits:
\begin{lstlisting}[basicstyle=\normalsize\ttfamily]
git diff  <commit> <commit> [<path>...]
\end{lstlisting}

\vspace*{1em}

\end{frame}

\begin{frame}[fragile]
\frametitle{Show versions from the past}
\begin{tcolorbox}[title=git show]
Shows Git objects (not only commits)
\begin{lstlisting}
SYNOPSIS
       git show [options] <object>...
\end{lstlisting}
\end{tcolorbox}
\vspace*{1em}
\textbf{Example:}
\begin{lstlisting}[basicstyle=\normalsize\ttfamily]
git show 9f529a:myfile.txt
\end{lstlisting}
\end{frame}

\begin{exerciseframe}
\frametitle{Exercise: log, diff, show \hfill (5 minutes)}
\begin{itemize}
  \item Create \texttt{green.txt} and \texttt{black.txt} similar to the 
  first two files.
  \item Add and commit them.
  \item Add a file \texttt{README} that describes the content of the 
  directory.
  \item Make changes to \texttt{red.txt} (remove the second line) and 
  show the difference with git diff.
  \item Add \texttt{red.txt} to the staging area and use \hlcommand{git 
  diff -{}-cached} to see what will be
        committed.
  \item Commit the change and and use the  \hlcommand{-v} switch to see 
  what's to
        be committed in your editor.
  \item Use \hlcommand{git log} to view the history.
  \item Use \hlcommand{git log -p red.txt} to see differences in 
  history\\
        for that particular file.
  \item Use \hlcommand{git show} to inspect a version of 
  \texttt{red.txt} from a previous
        commit.
  \item Use \hlcommand{git show} to get the old version back.
\end{itemize}
\end{exerciseframe}

%============================================================================
\section{Going back in time (checkout, reset)}

\begin{frame}[fragile]
\frametitle{Get older versions of a file back}
\begin{tcolorbox}[title=git checkout]
Checkout a branch or paths to the working tree
\end{tcolorbox}
\vspace*{0.5em}
\textbf{Example 1:}\\
You have unstaged changes and want to discard them:\\[0.5em]
\begin{lstlisting}[basicstyle=\normalsize\ttfamily]
git checkout myfile.txt
\end{lstlisting}
\vspace*{0.5em}
\textbf{Example 2: (only in emergency cases)}\\
Change the whole repository back to a certain commit\\[0.5em]
\begin{lstlisting}[basicstyle=\normalsize\ttfamily]
git checkout 9f529a
\end{lstlisting}
Warning: The command above leaves repo in 'detached HEAD' state!\\
More on \hlcommand{git checkout} in Part II when we discuss branching.
\end{frame}


\begin{frame}[fragile]
\frametitle{Revert changes from a single commit}
\begin{tcolorbox}[title=git revert]
Revert an existing commit and make a new commit for this.
\begin{lstlisting}
SYNOPSIS
       git revert <commit>
\end{lstlisting}
\end{tcolorbox}

\vspace*{1em}
\texttt{git revert} undoes a single commit -- it does not go
back to the previous state of a project by removing all subsequent 
commits.
In Git, this is actually called a reset (see next slide), not a revert.
\begin{center}
\includegraphics[width=0.3\textwidth]{./img/git-tutorial_changes-revert}
\end{center}
{\footnotesize \textbf{Picture taken from the very good Git tutorial from Atlassian: \url{https://www.atlassian.com/git/tutorials}}}
\end{frame}

\begin{frame}[fragile]
\frametitle{Reset the history}
\begin{tcolorbox}[title=git reset]
Reset current HEAD to the specified state
\begin{lstlisting}
SYNOPSIS
       git reset [-q] [<commit>] [--] <paths>...
\end{lstlisting}
\end{tcolorbox}
\vspace*{0.2em}
\textbf{Example 1:}\\ 
You have staged changes and want to unstage them:
\begin{lstlisting}
git reset <file>
\end{lstlisting}
\textbf{Example 2:}\\
You want the state of your repository back, i.e. from yesterday evening:
\begin{lstlisting}
git reset [-- hard] 9f529a
\end{lstlisting}
{\small \textbf{Warning}: When you do a reset like this, your history is lost
(all commit messages). You can rewrite history by this.\\
\textbf{When you use the (optional) \texttt{-{}-hard} switch your working tree is also reset!}}
\end{frame}

\begin{exerciseframe}
\frametitle{Exercise: checkout, reset, revert \hfill (5 minutes)}
\begin{itemize}
\item Step 1: Create two more files in your directory:\\
      \texttt{orange.txt} and \texttt{purple.txt}
\item Step 2: Commit the files to the repository
\item Step 3: Add second lines in \texttt{orange.txt} and 
\texttt{green.txt}\\
      (think about things that have these colors).
\item Step 4: Undo the changes in step 3 using \hlcommand{git reset}.
\item Step 5: Print out the last entry in the log.
\end{itemize}
\vspace*{0.5em}
\textbf{Alternative:}
\begin{itemize}
\item Redo step 3.
\item Use \hlcommand{git revert} to undo the changes instead of 
\hlcommand{git reset}.
\end{itemize}
\end{exerciseframe}


%============================================================================
\section{Ignore, move and remove files}

\begin{frame}[fragile]
\frametitle{Ignore files}
\begin{tcolorbox}[title=The .gitignore file]
Can be in the root or any sub directory of your Git repository.
\end{tcolorbox}
\vspace*{0.5em}
Example:
\begin{lstlisting}
$ cat .gitignore
*.aux
*.log
*.nav
*.out
*.snm
*.toc
*.vrb
git_intro.pdf
$
\end{lstlisting}
\vspace*{0.5em}
We have already seen another file that we could use!\\
(\texttt{./.git/info/exclude})
\end{frame}

\begin{frame}[fragile]
\frametitle{Move files within your repository}
\begin{tcolorbox}[title=git mv]
Put \texttt{git} in front of the mv command to make Git aware of the name change.
\end{tcolorbox}
\vspace*{0.5em}
Example:
\begin{lstlisting}
$ git mv file3 file5
$ git status
# On branch master
# Changes to be committed:
#   (use "git reset HEAD <file>..." to unstage)
#
#	renamed:    file3 -> file5
#
$ git commit -m "Renamed the file, because ..."
\end{lstlisting}
\end{frame}

\begin{frame}[fragile]
\frametitle{Remove files from version control}
\begin{tcolorbox}[title=git rm]
Put \texttt{git} in front of the rm command to make Git aware of the removal.\\[0.5em]
The command will remove the file from the working directory!\\
For \hlcommand{-{}-cached} option, see: \hlcommand{git help rm} 
\end{tcolorbox}
Example:
\begin{lstlisting}
$ git rm file5
rm 'file5'
$ git status
# On branch master
# Changes to be committed:
#   (use "git reset HEAD <file>..." to unstage)
#
#	deleted:    file5
#
$ git commit -m "Deleted old file. Not needed by the project any more."

\end{lstlisting}
\end{frame}


\begin{exerciseframe}
\frametitle{Exercise: ignore, rename and remove \hfill (5 minutes)}
\begin{itemize}
  \item Add a file \texttt{colors.log}. Do not stage it.
  \item Add a \texttt{.gitignore} file to ignore logfiles.
  \item Git will not care about \texttt{colors.log} any more\\
        (check with \hlcommand{git status}).
  \item The content of \texttt{.gitignore} may change over time.
  \item Why not put it into version control, too?\\
        \hlcommand{git add .gitignore; git commit -m 'Ignore logfiles.'}
  \item Rename and remove some of your (color) files.\\
        Do it with and without 'git' in front of the\\
       'mv' and 'rm' commands.\\
        What difference does \hlcommand{git status} show?
\end{itemize}
\end{exerciseframe}

%============================================================================
\section{Clone a project}

\begin{frame}[fragile]
\frametitle{Clone a project}
\begin{tcolorbox}[title=git clone]
Clones a repository from some other location.\\
Many protocols are supported. Despite to \hlcommand{git init} the cloned repo
may not be empty. Use \hlcommand{git log} to see the history.
\end{tcolorbox}
\vspace*{1em}
\textbf{Example:}\\
From Github\\[0.5em]
\begin{lstlisting}[basicstyle=\footnotesize\ttfamily]
git clone https://github.com/purepitch/git_course.git
\end{lstlisting}
\end{frame}

\begin{frame}[fragile]
\frametitle{Pull changes from a remote}
\begin{tcolorbox}[title=git pull]
Fetch from and integrate with another repository or a local 
branch.\\[0.5em]
Incorporates changes from a remote repository into the current branch. 
In its default mode, git pull is shorthand for \hlcommand{git fetch} 
followed by \hlcommand{git merge}.
\end{tcolorbox}
\vspace*{0.5em}
If you don't have much experience with branches (see Part II of this 
course) and just work on one branch (default: \texttt{master}) with 
one remote (default: \texttt{origin}), you can use \hlcommand{git pull}
for your convenience.\\[0.5em]
Sometimes after a pull Git will throw you into an editor with a merge 
commit message already prepared. Just save this one and commit the 
change locally. If there is a conflict Git cannot resolve itself, it 
will inform you about it and tell you what to do.
\end{frame}

\begin{frame}[fragile]
\frametitle{Push changes to a remote}
\begin{tcolorbox}[title=git push]
Update the remote server with changes you have made to your local 
repository.\\[0.5em]
You should always do a  \hlcommand{git pull} first, before you try to 
push. Make this a habit!
\end{tcolorbox}

At some point you might experience that your push gets rejected by the 
server. Don't worry! Try a \hlcommand{git pull} and read what Git 
tries to tell you. When all conflicts are resolved, the push should 
work.\\[0.5em]
If not: Contact your Git server administrator to check if you have 
indeed write permission to the repository.\\[0.5em]
More on \hlcommand{git pull} and \hlcommand{git push} in Part II.

\end{frame}

\begin{exerciseframe}
\frametitle{Exercise: clone, pull and push \hfill (10 minutes)}
\begin{itemize}
  \item Create an account on Github
  \item Fork the repository: \\
  \url{https://github.com/purepitch/git_course.git} \\
  (online via the web interface)
  \item Clone your fork to your local hard drive
  \item Make some changes and push them
  \item (optional) Give your neighbor write access to your repo
  \item (optional) Make changes, push and pull them.
  \item (optional) Create merge conflicts that will reject a push and 
  have to be resolved with a pull.
\end{itemize}
\end{exerciseframe}

%============================================================================
\section{References, Part II and Thank You}

\begin{frame}
\frametitle{Git commands covered today}
\begin{center}
    %\resizebox{\textwidth}{!}{
        \usetikzlibrary{positioning}

\begin{tikzpicture}
[
gitcommand/.style={rectangle, minimum width=6em, minimum height=2em, 
fill=blue!20, 
draw=blue, thick},
]

\node (git) at (0,0) {\texttt{git}};
\node [gitcommand, below of = git] (config) {\texttt{config}};
\node [gitcommand, left = of config] (help) {\texttt{help}};
\node [gitcommand, right = of config] (init) {\texttt{init}};
\node [gitcommand, below of = help] (status) {\texttt{status}};
\node [gitcommand, below of = config] (add) {\texttt{add}};
\node [gitcommand, below of = init] (commit) {\texttt{commit}};
\node [gitcommand, below of = status] (log) {\texttt{log}};
\node [gitcommand, below of = add] (diff) {\texttt{diff}};
\node [gitcommand, below of = commit] (show) {\texttt{show}};
\node [gitcommand, below of = log] (clone) {\texttt{clone}};
\node [gitcommand, below of = diff] (pull) {\texttt{pull}};
\node [gitcommand, below of = show] (push) {\texttt{push}};
\node [gitcommand, below of = clone] (checkout) {\texttt{checkout}};
\node [gitcommand, below of = pull] (reset) {\texttt{reset}};
\node [gitcommand, below of = push] (revert) {\texttt{revert}};
\node [gitcommand, below of = checkout] (mv) {\texttt{mv}};
\node [gitcommand, below of = reset] (rm) {\texttt{rm}};

%  \item \texttt{clone}

\end{tikzpicture}
    %}
\end{center}
\end{frame}

\begin{frame}
\frametitle{For further study}
\begin{itemize}
  \item Git reference site:\\
  \url{http://gitref.org}
  \item Website of the Git project:\\
  \url{http://git-scm.com}\\
  Here you can find the very good Pro Git book (PDF):\\
  \url{http://git-scm.com/book}
  \item Git introduction (Katy Huff)\\
        \url{http://www.youtube.com/watch?v=T0BE9ApIegc}
  % Many ideas for this talk were taken from Katy. Give credit!
  \item More advanced Git introduction (Scott Chacon)\\
        \url{http://www.youtube.com/watch?v=ZDR433b0HJY}\\
        (Scott explains the SHA1 hashes in more detail. Furthermore he shows
        how Git 'thinks' about version control compared to Subversion. He
        introduces branching very early - a must see talk!)
  % Scott Chacons Railsconf Git Talk: http://vimeo.com/1099027
  \item Tech Talk: Linus Torvalds on Git (very funny)\\
        \url{http://www.youtube.com/watch?v=4XpnKHJAok8}
  % Discussion on staged vs indexed vs cached:
  % http://git.661346.n2.nabble.com/Consistent-terminology-cached-staged-index-td6021438.html
\end{itemize}
\end{frame}

\begin{frame}
\frametitle{Part II: Advanced Git knowledge}
\begin{itemize}
  \item Git objects and hashes in detail
  \item Stage only parts of a file: patch mode for \texttt{git add} \\
  \hlcommand{git add -p}
  \item Work with (local) branches: create, delete and merge
  \item Resolve (merge) conflicts
  \item More on remotes: What happens when you clone?
  \item Fetch and merge remote branches, \hlcommand{git pull}
  \item \hlcommand{git stash}
  \item \hlcommand{git tag}
  \item \hlcommand{git rebase}
  \item Workflows in version control: distributed vs.~central
\end{itemize}
\end{frame}

\begin{frame}
\frametitle{Thank you}
\begin{textblock}{10}(3,-0.2)
  \includegraphics[width=0.75\textwidth]{./img/git-bingo.jpg}
\end{textblock}
\begin{textblock}{6}(10.2,1)
  \footnotesize
  \texttt{Comic source:}\\
  \url{http://geek-and-poke.com}
\end{textblock}
\end{frame}

%============================================================================

\end{document}

%% vim: shiftwidth=4 expandtab:
