%%%%%%%%%%%%%%%%%%%%%%%%%%%%%%%%%%%%%%%%%%%%%%%%%%%%%%%%%%%%%%%%%%%%%%%%%%%%
\begin{frame}[fragile]
\frametitle{Server access via SSH}
\begin{itemize}
\item All project members have an account on the version control
    server (\ttalert{host.name}).
\item Your username is usually your last name
\item Access to the server is only possible via the secure shell (\ttalert{ssh})
\item Access is only possible with an ssh key
\end{itemize}
\end{frame}

%%%%%%%%%%%%%%%%%%%%%%%%%%%%%%%%%%%%%%%%%%%%%%%%%%%%%%%%%%%%%%%%%%%%%%%%%%%%
\begin{frame}[fragile]
\frametitle{SSH Passphrase}

A passphrase (as opposed to a password) makes \ttalert{ssh}-access to the
server much easier.  After one has entered the passphrase, it is possible to
access the server multiple times without having to enter the passphrase
again.  It isn't even necessary to enter your password!  However, a
passphrase is much longer (and therefore much harder to crack):
\begin{lstlisting}
Mary had a little lamb, its fleece was white as snow.
\end{lstlisting}

The secure shell uses \emph{public key encryption} which means that we need
to copy our public key to the remote server and then we can open an
encrypted connection.  We can then send data to and receive data from the
server over this connection.
\end{frame}

%%%%%%%%%%%%%%%%%%%%%%%%%%%%%%%%%%%%%%%%%%%%%%%%%%%%%%%%%%%%%%%%%%%%%%%%%%%%
\begin{frame}[fragile]
\frametitle{SSH Passphrase (cont.)}

Yes, but why?

\ttt{ssh} is a very good way to send data over a network.  It's a standard
method and it's secure.  It is, however, very annoying to have to enter
one's password each time a connection is made to the server, so a passphrase
makes data transfers and server connections transparent and easy.  Using a
passphrase with an encrypted connection makes using Subversion (and Git)
much, much easier.

To generate your private and public keys on Windows, you first need to
install PuTTY.
\end{frame}

% vim: expandtab shiftwidth=4:
