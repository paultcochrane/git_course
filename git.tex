\section{Git}

\begin{frame}
\begin{center}
\vspace{5em}
\Huge GIT
\end{center}
\end{frame}
%%%%%%%%%%%%%%%%%%%%%%%%%%%%%%%%%%%%%%%%%%%%%%%%%%%%%%%%%%%%%%%%%%%%%%%%%%%%
\subsection{History}
\begin{frame}{Git--History}
\begin{itemize}
    \item Originally written for Linux kernel development.
    \item All Linux kernel developers used to be able to use the proprietary
	``Bitkeeper'' version control system for free.
    \item In 2005 there were further restrictions put on Bitkeeper so that
	it wasn't as free as it used to be.
    \item Linux Torvalds wasn't happy with the situation and so wrote his
	own system.
\end{itemize}
\end{frame}

%%%%%%%%%%%%%%%%%%%%%%%%%%%%%%%%%%%%%%%%%%%%%%%%%%%%%%%%%%%%%%%%%%%%%%%%%%%%
\begin{frame}{Goals of Git development}
\begin{itemize}
    \item Distributed development
    \item Scalable up to thousands of developers
    \item Fast and efficient
    \item Maintain integrity and privacy
    \item Enforced responsibility
    \item Immutable objects
    \item Atomic operations
    \item Support and promote development with branches
    \item Complete repositories
    \item Clean design
    \item Free as in freedom
\end{itemize}
\end{frame}

%%%%%%%%%%%%%%%%%%%%%%%%%%%%%%%%%%%%%%%%%%%%%%%%%%%%%%%%%%%%%%%%%%%%%%%%%%%%
\subsection{Concepts and help}

\begin{frame}[fragile]
\frametitle{Concepts}
\begin{itemize}
    \item Basically a versioned file system
    \item The version control system is developed on top of this
    \item The working copy is the repository (c.f. Subversion)
    \item 130+ commands
    \item ``porcelain'' and ''plumbing'' commands
    \item Distributed repository model can be used centrally
    (e.g. \href{http://github.com/}{\alert{Github}})
    \item Very fast!
\end{itemize}

First, we need to install Git:
\begin{lstlisting}
$ aptitude install git git-doc
\end{lstlisting}

Git on Windows
\begin{itemize}
    \item Cygwin: \url{http://www.cygwin.com}
    \item msysGit: \url{http://code.google.com/p/msysgit}
\end{itemize}

\end{frame}

%%%%%%%%%%%%%%%%%%%%%%%%%%%%%%%%%%%%%%%%%%%%%%%%%%%%%%%%%%%%%%%%%%%%%%%%%%%%
\begin{frame}[fragile]
\frametitle{SHA1-Hashes}
\begin{itemize}
    \item SHA1-Hash specifies \emph{the entire repository state} at the time
	of the commit
    \item Normally one only needs the first part in order to specify a given
	commit, e.g. \ttalert{f011c01}
    \item Enforces integrity and responsibility
\end{itemize}

\end{frame}

%%%%%%%%%%%%%%%%%%%%%%%%%%%%%%%%%%%%%%%%%%%%%%%%%%%%%%%%%%%%%%%%%%%%%%%%%%%%
\begin{frame}[fragile]
\frametitle{Blobs, Trees and Commits}
\begin{description}
\item[blob] \enquote{Binary large objects}; basically just files
\item[tree] Trees of files
\item[commit] Versioned state of a tree
\item[tags] Human-readable name of a commit
\end{description}
TODO: need a picture of these things
\end{frame}

%%%%%%%%%%%%%%%%%%%%%%%%%%%%%%%%%%%%%%%%%%%%%%%%%%%%%%%%%%%%%%%%%%%%%%%%%%%%
\begin{frame}[fragile]
\frametitle{Getting help from Git}
\vspace*{-3mm}
\begin{lstlisting}[basicstyle=\tiny\ttfamily\color{black}]
$ git --help
usage: git [--version] [--exec-path[=<path>]] [--html-path] [--man-path] [--info-path]
           [-p|--paginate|--no-pager] [--no-replace-objects] [--bare]
           [--git-dir=<path>] [--work-tree=<path>] [--namespace=<name>]
           [-c name=value] [--help] <command> [<args>]

The most commonly used git commands are:
   add        Add file contents to the index
   bisect     Find by binary search the change that introduced a bug
   branch     List, create, or delete branches
   checkout   Checkout a branch or paths to the working tree
   clone      Clone a repository into a new directory
   commit     Record changes to the repository
   diff       Show changes between commits, commit and working tree, etc
   fetch      Download objects and refs from another repository
   grep       Print lines matching a pattern
   init       Create an empty git repository or reinitialize an existing one
   log        Show commit logs
   merge      Join two or more development histories together
   mv         Move or rename a file, a directory, or a symlink
   pull       Fetch from and merge with another repository or a local branch
   push       Update remote refs along with associated objects
   rebase     Forward-port local commits to the updated upstream head
   reset      Reset current HEAD to the specified state
   rm         Remove files from the working tree and from the index
   show       Show various types of objects
   status     Show the working tree status
   tag        Create, list, delete or verify a tag object signed with GPG

See 'git help <command>' for more information on a specific command.
\end{lstlisting}
\end{frame}

%%%%%%%%%%%%%%%%%%%%%%%%%%%%%%%%%%%%%%%%%%%%%%%%%%%%%%%%%%%%%%%%%%%%%%%%%%%%
\begin{frame}[fragile]
\frametitle{Getting help from Git}

On Linux, all Git commands are available via \ttalert{man} pages.  One calls
the command like so:
\begin{itemize}
    \item \ttalert{man git init}
    \item \ttalert{man git add}
    \item \ttalert{man git commit}
\end{itemize}

Using the Git \ttalert{help} command also displays the appropriate
\ttt{man} page:
\begin{itemize}
    \item \ttalert{git help init}
    \item \ttalert{git help add}
    \item \ttalert{git help commit}
\end{itemize}
\end{frame}

%%%%%%%%%%%%%%%%%%%%%%%%%%%%%%%%%%%%%%%%%%%%%%%%%%%%%%%%%%%%%%%%%%%%%%%%%%%%
\subsection{Get started quickly!}
\begin{frame}[fragile]
\frametitle{Get started quickly---importing a new project}
Download the \ttalert{hallo} project and decompress it
\begin{lstlisting}
$ wget http://host.name/course_material/hallo.tar.gz
$ tar -xvzf hallo.tar.gz
\end{lstlisting}

Create a new repository: \ttalert{git init}
\begin{lstlisting}
$ cd hallo
$ git init
Initialized empty Git repository in /home/cochrane/hallo/.git/
\end{lstlisting}
That's it!

We don't need a server, nor do we need a network connection.  We can
happily sit nice and relaxed in a cafe, slurp on a coffee and work!
\end{frame}

%%%%%%%%%%%%%%%%%%%%%%%%%%%%%%%%%%%%%%%%%%%%%%%%%%%%%%%%%%%%%%%%%%%%%%%%%%%%
\begin{frame}[fragile]
\frametitle{Get started quickly---importing a new project (cont.)}
We don't have anything yet in our repository, however that isn't a problem.
We can have a look at the state of our working copy:

\ttalert{git status}
\begin{lstlisting}
$ git status
# On branch master
#
# Initial commit
#
# Untracked files:
#   (use "git add <file>..." to include in what will be committed)
#
#       Makefile
#       hallo.c
nothing added to commit but untracked files present (use "git add" to track)
\end{lstlisting}
Git gives us a hint as to what to do next.
\end{frame}

%%%%%%%%%%%%%%%%%%%%%%%%%%%%%%%%%%%%%%%%%%%%%%%%%%%%%%%%%%%%%%%%%%%%%%%%%%%%
\begin{frame}[fragile]
\frametitle{Get started quickly---importing a new project (cont.)}

In order to add files to the reposiotry, one needs to add them first:

\ttalert{git add <filename>}

To add all files in a directory, just specify the current directory:

\ttalert{git add .}
\begin{lstlisting}[basicstyle=\tiny\ttfamily\color{black}]
$ git add .
$ git status
# On branch master
#
# Initial commit
#
# Changes to be committed:
#   (use "git rm --cached <file>..." to unstage)
#
#       new file: Makefile
#       new file: hallo.c
\end{lstlisting}
The files have now been added to the \emph{index}, however are not yet
actually in the repository.  One says, that the files have been
\emph{staged}.  This is useful since sometimes we wish to \emph{unstage}
files.  Git also tells us how we do this.

\end{frame}

%%%%%%%%%%%%%%%%%%%%%%%%%%%%%%%%%%%%%%%%%%%%%%%%%%%%%%%%%%%%%%%%%%%%%%%%%%%%
\begin{frame}[fragile]
\frametitle{Get started quickly---importing a new project (cont.)}
\vspace*{-3mm}
In this case we want to make a \emph{commit}.

\ttalert{git commit [<filename> ...]} oder \ttalert{git commit -a}

\begin{lstlisting}
$ git commit -a -m "Initial import of the hallo project"
Created initial commit f011c01: Initial import of the hallo project
 2 files changed, 16 insertions(+), 0 deletions(-)
 create mode 100644 Makefile
 create mode 100644 hallo.c
\end{lstlisting}

Note:
\begin{itemize}
    \item The first line of the commit message is shown
    \item There are no revision numbers (c.f.~Subversion); commits are
	specified via SHA1 hashes.  The first part of the SHA1 hash is also
	shown: \ttalert{f011c01}.  This can be used as a reference to
	the commit.
\item File modes are also shown.  644 means:
    \begin{itemize}
    \item readable and writeable for the file owner (6)
    \item readable for the file group (4)
    \item readable for all other users (4)
    \end{itemize}
\end{itemize}
\end{frame}

%%%%%%%%%%%%%%%%%%%%%%%%%%%%%%%%%%%%%%%%%%%%%%%%%%%%%%%%%%%%%%%%%%%%%%%%%%%%
\begin{frame}
    \frametitle{Recommendations for commits}
    \begin{itemize}
	\item A commit should contain a single, self contained idea.
	\item The commit message should contain a short description of the
	    idea or change being made in this commit.
	    \begin{itemize}
		\item A one line \emph{subject line}
		\item An optional text describing more details of the
		    change.  The \emph{why} of the change is important.
		    Remember: this is a communication exercise.
	    \end{itemize}
	\item Commits should be as small as possible (\emph{atomic
	    commits}).
    \end{itemize}
\end{frame}

%%%%%%%%%%%%%%%%%%%%%%%%%%%%%%%%%%%%%%%%%%%%%%%%%%%%%%%%%%%%%%%%%%%%%%%%%%%%
\begin{frame}[fragile]
    \frametitle{What does \ttalert{git init} do?}
    \begin{itemize}
	\item Creates a \ttalert{.git} directory
	\item Creates various configuration files, e.g.
	    \ttalert{.git/config}
	\item Sets up the repository; the repository is inside the
	    \ttalert{.git} directory
    \end{itemize}
\end{frame}

%%%%%%%%%%%%%%%%%%%%%%%%%%%%%%%%%%%%%%%%%%%%%%%%%%%%%%%%%%%%%%%%%%%%%%%%%%%%
\subsection{Configuration}
\begin{frame}[fragile]
\frametitle{Configuration}
\vspace*{-3mm}
\ttalert{git config <parameter> <value>}

Only for this repository
\begin{lstlisting}[basicstyle=\tiny\ttfamily\color{black}]
$ git config user.name "VCS Course 15"
$ git config user.email "fake@gmx.de"
\end{lstlisting}

Globally for all repositories (saved in \ttalert{\$HOME/.gitconfig})
\begin{lstlisting}[basicstyle=\tiny\ttfamily\color{black}]
$ git config --global user.name "Paul Cochrane"
$ git config --global user.email "cochrane@my.email.address"
\end{lstlisting}

\ttalert{git config ----list}
\vspace*{-2mm}
\begin{multicols}{2}
In the \ttalert{hallo} working copy:
\begin{lstlisting}[basicstyle=\tiny\ttfamily\color{black}]
$ git config --list
user.email=cochrane@my.email.address
user.name=Paul Cochrane
core.repositoryformatversion=0
core.filemode=true
core.bare=false
core.logallrefupdates=true
user.name=VCS Course 15
user.email=fake@gmx.de
\end{lstlisting}
\columnbreak
In the user's HOME directory:
\begin{lstlisting}[basicstyle=\tiny\ttfamily\color{black}]
$ git config --list
user.email=cochrane@my.email.address
user.name=Paul Cochrane
\end{lstlisting}
\end{multicols}

\end{frame}

%%%%%%%%%%%%%%%%%%%%%%%%%%%%%%%%%%%%%%%%%%%%%%%%%%%%%%%%%%%%%%%%%%%%%%%%%%%%
\subsection{Workflow}
\begin{frame}[fragile]
\frametitle{Workflow}

Preparation:
\begin{itemize}
    \item \ttalert{git init} (only once)
    \item \ttalert{git checkout}
    \item \ttalert{git config}
\end{itemize}

Main steps:
\begin{itemize}
    \item \ttalert{git add} (before committing a file)
    \item \ttalert{git mv}
    \item \ttalert{git status}
    \item \ttalert{git diff}
    \item \ttalert{git rm}
    \item \ttalert{git log}
    \item \ttalert{git show}
    \item \ttalert{git commit}
\end{itemize}

\end{frame}

%%%%%%%%%%%%%%%%%%%%%%%%%%%%%%%%%%%%%%%%%%%%%%%%%%%%%%%%%%%%%%%%%%%%%%%%%%%%
\begin{frame}[fragile]
\frametitle{Seeing what we've done}
\vspace*{-3mm}
Let's add more languages to \ttalert{hallo.c}, build the program and test
that everything still works.  What has been changed?
\ttalert{git diff}
\begin{lstlisting}[basicstyle=\tiny\ttfamily\color{black}]
$ git diff
diff --git a/hallo.c b/hallo.c
index 008ac22..df60c79 100644
--- a/hallo.c
+++ b/hallo.c
@@ -2,6 +2,9 @@
 #include <stdio.h>
 
 int main(void) {
-    printf("Hallo, Welt!\n");
+    // north
+    printf("Moin, moin!\n");
+    // sued
+    printf("Gudday, mate!\n");
     return(0);
 }
\end{lstlisting}

The difference is taken between the repository and the working copy.  More
specifically: between the \ttalert{master} branch and the working copy.  In
general: between the currently checked out branch and the working copy.

\end{frame}

%%%%%%%%%%%%%%%%%%%%%%%%%%%%%%%%%%%%%%%%%%%%%%%%%%%%%%%%%%%%%%%%%%%%%%%%%%%%
\begin{frame}[fragile]
\frametitle{Seeing what we've done (cont.)}

Let's check in our changes:
\begin{lstlisting}[basicstyle=\tiny\ttfamily\color{black}]
$ git commit -v hallo.c
Created commit 0d964bb: Added a southern 'hallo' variant
 1 files changed, 4 insertions(+), 1 deletions(-)
\end{lstlisting}
\begin{itemize}
    \item The SHA1 hash is different to that in the \ttalert{diff} because
	the commit has changed the repository's state and thus has
	a different hash.
    \item The \ttalert{-v} option shows us the \ttalert{diff} while we are
	editing the commit message.
\end{itemize}
\end{frame}

%%%%%%%%%%%%%%%%%%%%%%%%%%%%%%%%%%%%%%%%%%%%%%%%%%%%%%%%%%%%%%%%%%%%%%%%%%%%
\begin{frame}[fragile]
\frametitle{Seeing what we've done (cont.)}
\ttalert{git log}
\begin{lstlisting}[basicstyle=\tiny\ttfamily\color{black}]
$ git log
commit 0d964bbeceec6d461e8d98a1d6e844300d66d30e
Author: VCS Course 15 <fake@gmx.de>
Date:   Sun Sep 19 14:19:00 2010 +0200

    Added a southern 'hallo' variant

commit f011c01e07c19299df6e9db2c757e317c285a6b9
Author: Paul Cochrane <cochrane@my.email.address>
Date:   Sun Sep 19 12:19:42 2010 +0200

    Initial import of the hallo project
\end{lstlisting}

\end{frame}

%%%%%%%%%%%%%%%%%%%%%%%%%%%%%%%%%%%%%%%%%%%%%%%%%%%%%%%%%%%%%%%%%%%%%%%%%%%%
\begin{frame}[fragile]
\frametitle{Seeing what we've done (cont.)}
\ttalert{git diff <hash1> <hash2> [<filename>]}

\begin{multicols}{2}
\begin{lstlisting}[basicstyle=\tiny\ttfamily\color{black}]
$ git diff \
    0d964bbeceec6d461e8d98a1d6e844300d66d30e\
    f011c01e07c19299df6e9db2c757e317c285a6b9
diff --git a/hallo.c b/hallo.c
index df60c79..008ac22 100644
--- a/hallo.c
+++ b/hallo.c
@@ -2,9 +2,6 @@
 #include <stdio.h>
 
 int main(void) {
-    // north
-    printf("Moin, moin!\n");
-    // sued
-    printf("Gudday, mate!\n");
+    printf("Hallo, Welt!\n");
     return(0);
 }
\end{lstlisting}
\columnbreak
\begin{lstlisting}[basicstyle=\tiny\ttfamily\color{black}]
$ git diff 0d964bb f011c01
diff --git a/hallo.c b/hallo.c
index df60c79..008ac22 100644
--- a/hallo.c
+++ b/hallo.c
@@ -2,9 +2,6 @@
 #include <stdio.h>
 
 int main(void) {
-    // north
-    printf("Moin, moin!\n");
-    // south
-    printf("Gudday, mate!\n");
+    printf("Hallo, Welt!\n");
     return(0);
 }
\end{lstlisting}
\end{multicols}
\end{frame}

%%%%%%%%%%%%%%%%%%%%%%%%%%%%%%%%%%%%%%%%%%%%%%%%%%%%%%%%%%%%%%%%%%%%%%%%%%%%
\begin{frame}[fragile]
\frametitle{Showing what we've got}
\ttalert{git show [<blob>|<tree>|<commit>|<tag>]}

\begin{multicols}{2}
\begin{lstlisting}[basicstyle=\tiny\ttfamily\color{black}]
$ git show hallo.c
commit 0d964bbeceec6d461e8d98a1d6e844300d66d30e
Author: VCS Course 15 <fake@gmx.de>
Date:   Sun Sep 19 14:19:00 2010 +0200

    Added a southern 'hallo' variant

diff --git a/hallo.c b/hallo.c
index 008ac22..df60c79 100644
--- a/hallo.c
+++ b/hallo.c
@@ -2,6 +2,9 @@
 #include <stdio.h>
 
 int main(void) {
-    printf("Hallo, Welt!\n");
+    // north
+    printf("Moin, moin!\n");
+    // south
+    printf("Gudday, mate!\n");
     return(0);
 }
\end{lstlisting}
\columnbreak
\begin{lstlisting}[basicstyle=\tiny\ttfamily\color{black}]
$ git show master
commit 0d964bbeceec6d461e8d98a1d6e844300d66d30e
Author: VCS Course 15 <fake@gmx.de>
Date:   Sun Sep 19 14:19:00 2010 +0200

    Added a southern 'hallo' variant

diff --git a/hallo.c b/hallo.c
index 008ac22..df60c79 100644
--- a/hallo.c
+++ b/hallo.c
@@ -2,6 +2,9 @@
 #include <stdio.h>
 
 int main(void) {
-    printf("Hallo, Welt!\n");
+    // north
+    printf("Moin, moin!\n");
+    // south
+    printf("Gudday, mate!\n");
     return(0);
 }
\end{lstlisting}
\end{multicols}

\end{frame}

%%%%%%%%%%%%%%%%%%%%%%%%%%%%%%%%%%%%%%%%%%%%%%%%%%%%%%%%%%%%%%%%%%%%%%%%%%%%
\begin{frame}[fragile]
\frametitle{Showing what we've got (cont.)}
\ttalert{git show [<blob>|<tree>|<commit>|<tag>]}

\begin{multicols}{2}
\begin{lstlisting}[basicstyle=\tiny\ttfamily\color{black}]
$ git show master^
commit f011c01e07c19299df6e9db2c757e317c285a6b9
Author: Paul Cochrane <cochrane@my.email.address>
Date:   Sun Sep 19 12:19:42 2010 +0200

    Initial import of the hallo project

diff --git a/Makefile b/Makefile
new file mode 100644
index 0000000..0e42f68
--- /dev/null
+++ b/Makefile
@@ -0,0 +1,9 @@
+.PHONY: clean
+
+default: hallo
+
+hallo: hallo.c
+       $(CC) -o hallo $<
+
+clean:
+       rm -f hallo *.o
diff --git a/hallo.c b/hallo.c
new file mode 100644
index 0000000..008ac22
--- /dev/null
+++ b/hallo.c
@@ -0,0 +1,7 @@
+// hallo: ein Programm um "hallo!" zu sagen
+#include <stdio.h>
+
+int main(void) {
+    printf("Hallo, Welt!\n");
+    return(0);
+}
\end{lstlisting}
\end{multicols}

\end{frame}


%%%%%%%%%%%%%%%%%%%%%%%%%%%%%%%%%%%%%%%%%%%%%%%%%%%%%%%%%%%%%%%%%%%%%%%%%%%%
\begin{frame}[fragile]
    \frametitle{Specifying commits}

    \SaveVerb{caret}|^|
    \SaveVerb{tilde}|~|
    \begin{description}
	\item[\Verb|c82a22c39c|] SHA hash
	\item[\Verb|HEAD|] name of the commit
	\item[\Verb|HEAD|\UseVerb{caret}] parent of \Verb|HEAD|
	\item[\Verb|HEAD|\UseVerb{caret}\UseVerb{caret}]
	    grandparent of \Verb|HEAD|
	\item[\Verb|HEAD|\UseVerb{tilde}\Verb|2|]
	    also grandparent of \Verb|HEAD|
	\item[\Verb|HEAD|\UseVerb{tilde}\Verb|4|]
	    great-great grandparent of \Verb|HEAD|
    \end{description}

    also works with branch names:

    \begin{description}
	\item[\Verb|master|\UseVerb{caret}]
	    parent of tip of ``\Verb|master|'' branch
	\item[\Verb|master|\UseVerb{tilde}\Verb|4|]
	    great-great grandparent of tip of ``\Verb|master|'' branch
	\item[\Verb|experimental|] tip of ``\Verb|experimental|'' branch
    \end{description}
\end{frame}


%%%%%%%%%%%%%%%%%%%%%%%%%%%%%%%%%%%%%%%%%%%%%%%%%%%%%%%%%%%%%%%%%%%%%%%%%%%%
\begin{frame}[fragile]
\frametitle{Ignoring files}

To ignore files in Git one simply needs a \ttalert{.gitignore} file with
entries for the files one wishes to ignore.  It is possible to use regular
expressions and wildcards such as \ttalert{*.o} in order to ignore multiple
files.

\begin{lstlisting}[basicstyle=\tiny\ttfamily\color{black}]
$ make
cc -o hallo hallo.c
$ ls
hallo  hallo.c  Makefile
$ git status
# On branch master
# Untracked files:
#   (use "git add <file>..." to include in what will be committed)
#
#       hallo
#       hallo.c~
nothing added to commit but untracked files present (use "git add" to track)
$ vim .gitignore
$ git status
# On branch master
# Untracked files:
#   (use "git add <file>..." to include in what will be committed)
#
#       .gitignore
nothing added to commit but untracked files present (use "git add" to track)
\end{lstlisting}

\end{frame}

%%%%%%%%%%%%%%%%%%%%%%%%%%%%%%%%%%%%%%%%%%%%%%%%%%%%%%%%%%%%%%%%%%%%%%%%%%%%
\begin{frame}[fragile]
    \frametitle{Ignoring files (cont.)}

Because the \ttalert{.gitignore} file can change (and it's highly likely
that we're interested in these changes) one also keeps this file under
version control.  We therefore add the file and commit it:
\begin{lstlisting}
$ git add .gitignore
$ git commit .gitignore
Created commit 38d2e3c: Ignoring the hallo program and *~ files
 1 files changed, 2 insertions(+), 0 deletions(-)
 create mode 100644 .gitignore
\end{lstlisting}
\end{frame}

%%%%%%%%%%%%%%%%%%%%%%%%%%%%%%%%%%%%%%%%%%%%%%%%%%%%%%%%%%%%%%%%%%%%%%%%%%%%
\subsection{Branch, Merge, Tag}
\begin{frame}[fragile]
\frametitle{Showing branches}
\ttalert{git show-branch}
\begin{itemize}
\item shows the available branches
\item shows with an asterisk \ttalert{*} which is the current branch
\end{itemize}
\begin{lstlisting}
$ git show-branch --list
* [master] Ignoring the hallo program and *~ files
\end{lstlisting}

\begin{lstlisting}
git show-branch --more=35
[master] Ignoring the hallo program and *~ files
[master^] Added a southern 'hallo' variant
[master~2] Initial import of the hallo project
\end{lstlisting}
\end{frame}

%%%%%%%%%%%%%%%%%%%%%%%%%%%%%%%%%%%%%%%%%%%%%%%%%%%%%%%%%%%%%%%%%%%%%%%%%%%%
\begin{frame}[fragile]
    \frametitle{Showing branches (cont.)}
\ttalert{git branch} shows available branches and which branch is current
\begin{lstlisting}
$ git branch
* master
\end{lstlisting}
\ttalert{git branch -v} shows the commit hash and the commit message of the
most recent commit
\begin{lstlisting}
$ git branch -v
* master 38d2e3c Ignoring the hallo program and *~ files
\end{lstlisting}
\end{frame}

%%%%%%%%%%%%%%%%%%%%%%%%%%%%%%%%%%%%%%%%%%%%%%%%%%%%%%%%%%%%%%%%%%%%%%%%%%%%
\begin{frame}[fragile]
\frametitle{A new branch}
\ttalert{git branch <branchname> [<initial commit>]}
\begin{lstlisting}[basicstyle=\tiny\ttfamily\color{black}]
$ git branch hallo-perl
$ git branch -v
  hallo-perl 38d2e3c Ignoring the hallo program and *~ files
* master     38d2e3c Ignoring the hallo program and *~ files
$ git checkout hallo-perl
Switched to branch "hallo-perl"
$ git branch -v
* hallo-perl 38d2e3c Ignoring the hallo program and *~ files
  master     38d2e3c Ignoring the hallo program and *~ files
\end{lstlisting}
This could also have been done with \ttalert{git checkout -b hallo-perl}.

\begin{lstlisting}[basicstyle=\tiny\ttfamily\color{black}]
$ vim hallo.pl
$ chmod +x hallo.pl
$  ./hallo.pl 
Moin, moin!
Gudday, mate!
$ git add hallo.pl
$ git commit -m "Translated the hallo program into Perl" hallo.pl
Created commit 8e92aec: Translated the hallo program into Perl
 1 files changed, 8 insertions(+), 0 deletions(-)
 create mode 100755 hallo.pl
\end{lstlisting}
\end{frame}

%%%%%%%%%%%%%%%%%%%%%%%%%%%%%%%%%%%%%%%%%%%%%%%%%%%%%%%%%%%%%%%%%%%%%%%%%%%%
\begin{frame}[fragile]
\frametitle{A new branch (cont.)}

\begin{lstlisting}
$ git branch -v
* hallo-perl 8e92aec Translated the hallo program into Perl
  master     38d2e3c Ignoring the hallo program and *~ files
$ git show-branch --more=10
* [hallo-perl] Translated the hallo program into Perl
 ! [master] Ignoring the hallo program and *~ files
--
*  [hallo-perl] Translated the hallo program into Perl
*+ [master] Ignoring the hallo program and *~ files
*+ [master^] Added a southern 'hallo' variant
*+ [master~2] Initial import of the hallo project
\end{lstlisting}

One can display the branches graphically with \ttalert{gitk} or with
\ttalert{tig}
\begin{lstlisting}
$ aptitude install gitk
$ gitk
$
$ aptitude install tig
$ tig
\end{lstlisting}
\end{frame}

%%%%%%%%%%%%%%%%%%%%%%%%%%%%%%%%%%%%%%%%%%%%%%%%%%%%%%%%%%%%%%%%%%%%%%%%%%%%
\begin{frame}[fragile]
\frametitle{A new branch (cont.)}

We work on \ttalert{hallo.pl} a bit more\ldots
\begin{lstlisting}
$ git diff hallo-perl^ hallo-perl
diff --git a/hallo.pl b/hallo.pl
index 411c241..9abe587 100755
--- a/hallo.pl
+++ b/hallo.pl
@@ -4,5 +4,11 @@
 use warnings;
 use strict;
 
-print "Moin, moin!\n";
-print "Gudday, mate!\n";
+my %greetings = (
+    "north" => "Moin, moin!",
+    "south" => "Gudday, mate!",
+);
+
+for my $greeting ( keys $greetings ) {
+    print $greeting, "\n";
+}
\end{lstlisting}
\end{frame}

%%%%%%%%%%%%%%%%%%%%%%%%%%%%%%%%%%%%%%%%%%%%%%%%%%%%%%%%%%%%%%%%%%%%%%%%%%%%
\begin{frame}[fragile]
\frametitle{A new branch (cont.)}

What do the branches look like now?
\begin{lstlisting}[basicstyle=\tiny\ttfamily\color{black}]
$ git show-branch --more=10
* [hallo-perl] The hallo Perl version is now more flexible
 ! [master] Ignoring the hallo program and *~ files
--
*  [hallo-perl] The hallo Perl version is now more flexible
*  [hallo-perl^] Translated the hallo program into Perl
*+ [master] Ignoring the hallo program and *~ files
*+ [master^] Added a southern 'hallo' variant
*+ [master~2] Initial import of the hallo project
\end{lstlisting}

And, just because we can\ldots:
\begin{lstlisting}[basicstyle=\tiny\ttfamily\color{black}]
$ git diff master^ hallo-perl
....
$ git diff hallo-perl^ master~2
....
\end{lstlisting}

Work on the master branch a bit more\ldots
 
\end{frame}

%%%%%%%%%%%%%%%%%%%%%%%%%%%%%%%%%%%%%%%%%%%%%%%%%%%%%%%%%%%%%%%%%%%%%%%%%%%%
\begin{frame}[fragile]
\frametitle{Merge back into the master branch}

Our work in the new branch is now finished.  We now want to merge the
changes back into the master branch (a.k.a. trunk).

Change back to the master branch
\begin{lstlisting}[basicstyle=\tiny\ttfamily\color{black}]
$ git checkout master
Switched to branch "master"
\end{lstlisting}

Check that everything is clean (otherwise problems can turn up more easily
when the merge is performed)
\begin{lstlisting}[basicstyle=\tiny\ttfamily\color{black}]
$ git status
# On branch master
nothing to commit (working directory clean)
\end{lstlisting}

And finally merge!
\begin{lstlisting}[basicstyle=\tiny\ttfamily\color{black}]
$ git merge hallo-perl
Updating 38d2e3c..4738673
Fast forward
 hallo.pl |   14 ++++++++++++++
 1 files changed, 14 insertions(+), 0 deletions(-)
 create mode 100755 hallo.pl
\end{lstlisting}
\end{frame}

%%%%%%%%%%%%%%%%%%%%%%%%%%%%%%%%%%%%%%%%%%%%%%%%%%%%%%%%%%%%%%%%%%%%%%%%%%%%
\begin{frame}[fragile]
\frametitle{Oops!}

We work on the master branch a bit more
\begin{lstlisting}[basicstyle=\tiny\ttfamily\color{black}]
$ git commit -a
Created commit 96e56ce: Added another language
 1 files changed, 2 insertions(+), 0 deletions(-)
cochrane@nb206ptc:~/hallo$ vim hallo.c
cochrane@nb206ptc:~/hallo$ make
cc -o hallo hallo.c
cochrane@nb206ptc:~/hallo$ ./hallo
Moin, moin!
Gruess Gott!
Gudday, mate!
Arf, arf!
cochrane@nb206ptc:~/hallo$ git commit -a
Created commit b3fc5cb: The languages are now even more southern
 1 files changed, 2 insertions(+), 0 deletions(-)
\end{lstlisting}
\end{frame}

%%%%%%%%%%%%%%%%%%%%%%%%%%%%%%%%%%%%%%%%%%%%%%%%%%%%%%%%%%%%%%%%%%%%%%%%%%%%
\begin{frame}[fragile]
\frametitle{Oops! (cont.)}

Now do everything in Perl and find errors
\begin{lstlisting}[basicstyle=\tiny\ttfamily\color{black}]
cochrane@nb206ptc:~/hallo$ git checkout hallo-perl
Switched to branch "hallo-perl"
cochrane@nb206ptc:~/hallo$ vim hallo.pl
cochrane@nb206ptc:~/hallo$ git commit -a
Created commit 55686f4: Further development
 1 files changed, 1 insertions(+), 0 deletions(-)
cochrane@nb206ptc:~/hallo$ vim hallo.pl
cochrane@nb206ptc:~/hallo$ vim hallo.pl
cochrane@nb206ptc:~/hallo$ git commit -a
Created commit 869aa90: Added another language
 1 files changed, 2 insertions(+), 1 deletions(-)
cochrane@nb206ptc:~/hallo$ ./hallo.pl
Global symbol "$greetings" requires explicit package name at ./hallo.pl line
14.
Type of arg 1 to keys must be hash (not scalar dereference) at ./hallo.pl
line 14, near "$greetings ) "
Execution of ./hallo.pl aborted due to compilation errors.
cochrane@nb206ptc:~/hallo$ vim hallo.pl
$ ./hallo.pl
Gudday, mate!
Quark, quark!
Arf, arf!
Moin, moin!
\end{lstlisting}
We've corrected the error!
\end{frame}

%%%%%%%%%%%%%%%%%%%%%%%%%%%%%%%%%%%%%%%%%%%%%%%%%%%%%%%%%%%%%%%%%%%%%%%%%%%%
\begin{frame}[fragile]
\frametitle{Oops! (cont.)}

Quickly check in the working version\ldots
\begin{lstlisting}[basicstyle=\tiny\ttfamily\color{black}]
cochrane@nb206ptc:~/hallo$ git diff
diff --git a/hallo.pl b/hallo.pl
index ed0b3a5..3f08db0 100755
--- a/hallo.pl
+++ b/hallo.pl
@@ -11,6 +11,6 @@ my %greetings = (
     "even_southerner" => "Arf, arf!",
 );
 
-for my $greeting ( keys %greetings ) {
-    print $greeting, "\n";
+for my $location ( keys %greetings ) {
+    print $greetings{$location}, "\n";
 }
cochrane@nb206ptc:~/hallo$ git commit -a
Created commit a71053f: Corrected major errors
 1 files changed, 2 insertions(+), 2 deletions(-)
\end{lstlisting}
\end{frame}

%%%%%%%%%%%%%%%%%%%%%%%%%%%%%%%%%%%%%%%%%%%%%%%%%%%%%%%%%%%%%%%%%%%%%%%%%%%%
\begin{frame}[fragile]
\frametitle{Oops! (cont.)}
\vspace*{-3mm}
A new merge with the master branch
\begin{lstlisting}[basicstyle=\tiny\ttfamily\color{black}]
cochrane@nb206ptc:~/hallo$ git checkout master
Switched to branch "master"
cochrane@nb206ptc:~/hallo$ git merge hallo-perl
Merge made by recursive.
 hallo.pl |    6 ++++--
 1 files changed, 4 insertions(+), 2 deletions(-)
\end{lstlisting}
\vspace*{-3mm}
\ttalert{git log ----graph} is cool
\begin{lstlisting}[basicstyle=\tiny\ttfamily\color{black}]
cochrane@nb206ptc:~/hallo$ git log --graph --pretty=oneline --abbrev-commit
*   3067f40... Merge branch 'hallo-perl'
|\  
| * a71053f... Corrected major errors
| * 869aa90... Added another language
| * 55686f4... Further development
* | b3fc5cb... The languages are now even more southern
* | 96e56ce... Added another language
|/  
* 4738673... The hallo Perl version is now more flexible
* 8e92aec... Translated the hallo program into Perl
* 38d2e3c... Ignoring the hallo program and *~ files
* 0d964bb... Added a southern 'hallo' variant
* f011c01... Initial import of the hallo project
\end{lstlisting}
\vspace*{-3mm}
\ttalert{gitk} and \ttalert{tig} are too
\end{frame}

%%%%%%%%%%%%%%%%%%%%%%%%%%%%%%%%%%%%%%%%%%%%%%%%%%%%%%%%%%%%%%%%%%%%%%%%%%%%
\begin{frame}{Exercise---Work with a Git repository}

\begin{itemize}
    \item Get a pre-existing project (e.g. a Sourceforge or GitHub project)
    \item Make a Git repository
    \item Add the files to the repo and check them in
    \item Change files and commit them
    \item Make a branch (or branches) and change to it (them)
    \item Change things in the branch and commit them
    \item Merge the changes back into the master branch
    \item Delete a branch (or branches)
    \item Use these commands: \ttalert{git log}, \ttalert{git status}, \ttalert{git diff}
\end{itemize}

\end{frame}

%%%%%%%%%%%%%%%%%%%%%%%%%%%%%%%%%%%%%%%%%%%%%%%%%%%%%%%%%%%%%%%%%%%%%%%%%%%%
\subsection{Working with others}
\begin{frame}{Working with others (also online)}
\begin{description}
\item[\ttalert{git clone}] Clones a pre-existing repository; basically copy
    into a new directory and get the current branch
\item[\ttalert{git pull}] Pull changes from another repo into the local
    repo
\item[\ttalert{git push}] Push changes from a local repo into another
\item[\ttalert{git format-patch}] Generate a patch file, which one can then
    send as an email attachment
\end{description}
\end{frame}

%%%%%%%%%%%%%%%%%%%%%%%%%%%%%%%%%%%%%%%%%%%%%%%%%%%%%%%%%%%%%%%%%%%%%%%%%%%%
\subsection{Extras}
\begin{frame}[fragile]
\frametitle{Extras}

\ttalert{git stash}

\begin{itemize}
\item Similar to the \enquote{changelist} concept in Subversion
\item One is in the middle of a task
\item Something unrelated needs to be fixed quickly
\item Used \ttalert{git stash} to save the current state (or more explicitly:
    \ttalert{git stash save}
\item The working copy is now \enquote{clean}
\item One is now able to work on the new topic and check in the changes
\item \ttalert{git stash pop} reinstates the old state
\item Now one can work on the previous task again, where one left off
\item \ttalert{git stash} works like a stack (push/pop operations)
\end{itemize}

\end{frame}

%%%%%%%%%%%%%%%%%%%%%%%%%%%%%%%%%%%%%%%%%%%%%%%%%%%%%%%%%%%%%%%%%%%%%%%%%%%%
\begin{frame}[fragile]
\frametitle{Extras (cont.)}

\begin{description}
\item[\ttalert{git stash list}] List all \enquote{stashed} states
\item[\ttalert{git stash show}] Show a stashed state
\item[\ttalert{git stash apply}] Apply a state to the current working copy
\item[\ttalert{git stash pop}] Pop the \enquote{top most} state off the
    stack and apply it onto the current working copy
\end{description}

%\begin{itemize}
%\item rebase
%\end{itemize}

\end{frame}

%%%%%%%%%%%%%%%%%%%%%%%%%%%%%%%%%%%%%%%%%%%%%%%%%%%%%%%%%%%%%%%%%%%%%%%%%%%%
\subsection{Git and Subversion}

\begin{frame}[fragile]
\frametitle{Git and Subversion repostiories}

Git also plays nicely with other version control systems.  For instance CVS,
Arch, and thankfully, Subversion.  For this one needs to install the
\ttalert{git-svn} package:
\begin{lstlisting}
$ aptitude install git-svn
\end{lstlisting}

The most important commands are:
\begin{description}
    \item[\ttalert{git svn clone}] Check out a Subversion repository into a
	local Git repository (including the complete history of the
	Subversion repo)
    \item[\ttalert{git svn fetch}] Fetch new revisions from the Subversion
	repository but don't apply them to the local repository
    \item[\ttalert{git svn rebase}] Fetch new revisions from the Subversion
	repository, apply them to the local Git repository and move the
	current commit to the most recent commit
    \item[\ttalert{git svn log}] Get the Subversion log (one can also simply
	use \ttalert{git log} in the local Git repository)
    \item[\ttalert{git svn dcommit}] Commit all local Git repository commits
	one-by-one back to the Subversion repository (\enquote{diff} commit)
\end{description}

\end{frame}

%%%%%%%%%%%%%%%%%%%%%%%%%%%%%%%%%%%%%%%%%%%%%%%%%%%%%%%%%%%%%%%%%%%%%%%%%%%%
\begin{frame}[fragile]
\frametitle{Clone a Subversion repository}
\ttalert{git svn clone <svn-repository-url>}
\begin{lstlisting}
$ git svn clone\
    svn+ssh://vcskurs15@host.name/svnroot/projects/hallo
Initialized empty Git repository in /home/cochrane/Kurs/hallo/.git/
    A   hallo.c
    A   Makefile
r1 = b423079e7a66c29b0890efec929a192f420ddc63 (git-svn)
    D   hallo.c
    D   Makefile
    A   trunk/hallo.c
    A   trunk/Makefile
W: +empty_dir: branches
W: +empty_dir: tags

....

r51 = 04468d313715388661ddc6b76b03d21d3b779fcc (git-svn)
Checked out HEAD:
  svn+ssh://vcskurs15@host.name/svnroot/projects/hallo r51
\end{lstlisting}
\end{frame}

%%%%%%%%%%%%%%%%%%%%%%%%%%%%%%%%%%%%%%%%%%%%%%%%%%%%%%%%%%%%%%%%%%%%%%%%%%%%
\begin{frame}[fragile]
\frametitle{Do some work\ldots}
Now we just work with Git as per usual.
\begin{itemize}
\item \ttalert{git branch}
\item \ttalert{git checkout -b <branchname>}
\item \ttalert{git commit}
\end{itemize}

If we wants to use the work of others (either via Subversion directly with
the server or via \ttt{git-svn}) one just needs to run: \ttalert{git-svn rebase}

\begin{lstlisting}
$ git svn rebase
    M   trunk/hallo.py
r52 = d76aa8a9eae3f25e14b17c6cacea9be80da23b76 (git-svn)
First, rewinding head to replay your work on top of it...
Fast-forwarded master to refs/remotes/git-svn.
\end{lstlisting}
\end{frame}

%%%%%%%%%%%%%%%%%%%%%%%%%%%%%%%%%%%%%%%%%%%%%%%%%%%%%%%%%%%%%%%%%%%%%%%%%%%%
\begin{frame}[fragile]
\frametitle{Do some work\ldots (cont.)}

One can make commits in the local Git repository (one could be in a cafe or
a in the pub; it isn't necessary to be online!).

We now improve and extend \ttalert{hallo.py} and make some commits.
\begin{lstlisting}
$ git commit -a
Created commit 4a7a298: Code is now more understandable: the loop didn't go over regions
 1 files changed, 2 insertions(+), 2 deletions(-)
$ git commit -a
Created commit d8b5899: The region is now mentioned when one is greeted
 1 files changed, 1 insertions(+), 1 deletions(-)
\end{lstlisting}
\end{frame}

%%%%%%%%%%%%%%%%%%%%%%%%%%%%%%%%%%%%%%%%%%%%%%%%%%%%%%%%%%%%%%%%%%%%%%%%%%%%
\begin{frame}[fragile]
\frametitle{Do some work\ldots (cont.)}

As soon as we have network again and we're happy with what we've developed,
we can upload our changes to the server using \ttalert{git svn dcommit}.
This command plays all commits back to the server one-by-one and one after
the other.  It's a good idea to do a \ttalert{git svn rebase} beforehand
because it is likely that other people have checked in changes to the server
in between times.
\begin{lstlisting}[basicstyle=\tiny\ttfamily\color{black}]
$ git svn rebase
Current branch master is up to date.
$ git svn dcommit
Committing to svn+ssh://host.name/svnroot/projects/hallo ...
    M   trunk/hallo.py
Committed r53
    M   trunk/hallo.py
r53 = dd95db3b2f51612e042fca61431fc368f8b94006 (git-svn)
No changes between current HEAD and refs/remotes/git-svn
Resetting to the latest refs/remotes/git-svn
trunk/hallo.py: needs update
    M   trunk/hallo.py
Committed r54
    M   trunk/hallo.py
r54 = 75b7b7ebb20c4782dc9aff06bf3a6e54c1556249 (git-svn)
No changes between current HEAD and refs/remotes/git-svn
Resetting to the latest refs/remotes/git-svn
\end{lstlisting}

\end{frame}

%%%%%%%%%%%%%%%%%%%%%%%%%%%%%%%%%%%%%%%%%%%%%%%%%%%%%%%%%%%%%%%%%%%%%%%%%%%%
\subsection{Resources}
\begin{frame}{Resources}
\begin{itemize}
\item O'Reilly Git Book
\item Git Community Book: \url{http://book.git-scm.com}
\item Scott Chacons Railsconf Git Talk:
    \url{http://www.gitcasts.com/git-talk}
\item Wikipedia-Eintrag: \url{http://en.wikipedia.org/wiki/Git_(software)}
\item \ttt{git-svn} Tutorial:
    \url{http://trac.parrot.org/parrot/wiki/git-svn-tutorial}
\end{itemize}
\end{frame}


merging commits:  git rebase -i <commit>

splitting commits: git rebase -i <commit>
  - mark which commit to edit
  - when rebase reaches that commit use \lstinline{git reset HEAD^} to reset
  before the commit but keep your working tree intact
  - incrementally add changes and commit them (git add -p can be useful)
  - run git rebase --continue to proceed applying the commits after the
  now-split commit
  (adapted from
  \url{http://stackoverflow.com/questions/4307095/git-how-to-split-up-a-commit-buried-in-history})


reordering commits: git rebase -i <commit>

pick, squash, edit

http://www.gitready.com/advanced/2009/02/10/squashing-commits-with-rebase.html

http://www.gitready.com/advanced/2009/03/20/reorder-commits-with-rebase.html

git format-patch <anfangscommit>..<endcommit>


git config --global color.ui auto  --> automatically colour diffs, logs and
other git output

git diff --word-diff  --> do a diff on words and not on lines.  Per default
in colour, but if don't have a colour terminal minus signs (-) and plus
signs (+) are used around words to show what has been removed or added
respectively.

tig as an alternative to gitk

create bare repo -> clone -> make commits in local repo -> git push -> error
-> why?  A: because there isn't a branch on the server, there needs to be a
connection between the local "master" branch and the "origin" on the server,
hence one needs to run "git push origin master" in order to make this
connection.
