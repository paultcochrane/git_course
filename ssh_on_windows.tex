%%%%%%%%%%%%%%%%%%%%%%%%%%%%%%%%%%%%%%%%%%%%%%%%%%%%%%%%%%%%%%%%%%%%%%%%%%%%
\begin{frame}
    \frametitle{Install PuTTY}

    \begin{itemize}
        \item Download PuTTY from
            \url{http://www.chiark.greenend.org.uk/~sgtatham/putty/download.html}
        \item Use the link under:
            \enquote{A Windows installer for everything except PuTTYtel}
        \item Double click on the installer and choose the default options
            in each window which appears when installing the software.
    \end{itemize}
\end{frame}

%%%%%%%%%%%%%%%%%%%%%%%%%%%%%%%%%%%%%%%%%%%%%%%%%%%%%%%%%%%%%%%%%%%%%%%%%%%%
\begin{frame}[fragile]
    \frametitle{Configure PuTTY}

    \begin{itemize}
        \item PuTTY Key Generator from Windows \enquote{Start Menu}
        \item Click on \enquote{Generate}
        \item Enter the passphrase and confirm
        \item Click on \enquote{Save private key}; save key as \ttt{host.ppk}
        \item Send the \enquote{Public key for pasting into OpenSSH
            authorized\_keys file} to \url{cochrane@}
    \end{itemize}
\end{frame}

%%%%%%%%%%%%%%%%%%%%%%%%%%%%%%%%%%%%%%%%%%%%%%%%%%%%%%%%%%%%%%%%%%%%%%%%%%%%
\begin{frame}[fragile]
    \frametitle{Configure PuTTY (cont.)}

    \begin{itemize}
        \item Start PuTTY from the Windows \enquote{Start Menu}
        \item Session $\rightarrow$
            Host name: \ttt{host.name}
        \item Connection $\rightarrow$ Data $\rightarrow$ Auto-login
            username: \ttt{your username}
        \item Connection $\rightarrow$ SSH $\rightarrow$ Auth $\rightarrow$
            Private key for authentication: \ttt{host.ppk}
        \item Session: Save
    \end{itemize}
\end{frame}

%%%%%%%%%%%%%%%%%%%%%%%%%%%%%%%%%%%%%%%%%%%%%%%%%%%%%%%%%%%%%%%%%%%%%%%%%%%%
\begin{frame}[fragile]
    \frametitle{Configure PuTTY (cont.)}

    \begin{itemize}
        \item Start \enquote{Pageant}
        \item Pageant can also be started at login time by adding a link to
            it in the user's \enquote{Startup} folder.
        \item Right-click on the Pageant icon $\rightarrow$ Add key
        \item Select \ttt{host.ppk} and type your passphrase (you only need
            to do this once every time you log on to Windows)
        \item Right-click on the Pageant icon $\rightarrow$ Saved sessions
            $\rightarrow$ \ttt{host.name}
        \item A console should appear with a prompt with your username and
            the name of the subversion server in it.
    \end{itemize}
\end{frame}

%%%%%%%%%%%%%%%%%%%%%%%%%%%%%%%%%%%%%%%%%%%%%%%%%%%%%%%%%%%%%%%%%%%%%%%%%%%%
\begin{frame}
    \frametitle{Notepad++ installation}
    \begin{itemize}
	\item Download Notepad++ from \url{http://notepad-plus-plus.org}
	\item Click on Agree, Next ... Install
    \end{itemize}

    \begin{itemize}
        \item We now have enough infrastructure in place that we can start
            talking about version control
    \end{itemize}
\end{frame}

% vim: expandtab shiftwidth=4:
