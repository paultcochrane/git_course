\section{Cygwin installation}

%%%%%%%%%%%%%%%%%%%%%%%%%%%%%%%%%%%%%%%%%%%%%%%%%%%%%%%%%%%%%%%%%%%%%%%%%%%%
\begin{frame}
    \frametitle{Install Cygwin}

    \begin{itemize}
        \item Cygwin website: \url{http://www.cygwin.com}
        \item Download the \ttt{setup.exe} file
        \item Go to the folder containing this file and double click on it
        \item Click on \enquote{Next} a few times...
        \item Select \url{http://ftp.gwdg.de} as the download mirror
    \end{itemize}
\end{frame}

%%%%%%%%%%%%%%%%%%%%%%%%%%%%%%%%%%%%%%%%%%%%%%%%%%%%%%%%%%%%%%%%%%%%%%%%%%%%
\begin{frame}
    \frametitle{Install Cygwin (cont.)}

    \begin{itemize}
        \item Install the following packages:
            \begin{itemize}
                \item Devel $\rightarrow$ autoconf, automake, doxygen,
                    gcc4-fortran, gcc4-g++, make, subversion
                \item Net $\rightarrow$ openssh
                \item Perl $\rightarrow$ perl
                \item Publishing $\rightarrow$ texlive,
                    texlive-collection-basic,
                    texlive-collection-fontsrecommended,
                    texlive-collection-latex,
                    texlive-collection-latexextra,
                    texlive-collection-latexrecommended,
                    texlive-collection-pictures
                \item Python $\rightarrow$ python, python-tkinter
                \item X11 $\rightarrow$ fvwm, xinit, xorg-docs, xorg-server
            \end{itemize}
        \item Resolve dependencies (click on \enquote{Next})
        \item Ignore the \enquote{pango} error at the end and let the
            installer create a desktop icon and start menu item.
        \item Start the Cygwin terminal to set up the environment properly
    \end{itemize}
\end{frame}

% vim: expandtab shiftwidth=4:
