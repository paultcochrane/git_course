\section{About the course}

\begin{frame}
\frametitle{Course goals}

At the end of this course you should:

\begin{itemize}
    \item feel comfortable using Git
    \item know where to get further help, if necessary
    \item be able to use Git on private projects
    \item be able to collaborate with others using remote repositories
\end{itemize}
\end{frame}

\begin{frame}
\frametitle{Course outline}
\begin{itemize}
    \item Introduction to Git and version control systems
    \item Installing Git
    \item Creating a first repository
    \item Getting help
    \item Tracking/staging/committing
    \item Configuring repositories
    \item General workflow
    \item Getting repository information
    \item Working with others
    \item Using branches and tags
    \item Rewriting history
    \item Contributing to Open Source projects
\end{itemize}
\end{frame}

% XXX: course flow as diagram

\begin{frame}
\frametitle{Course information}
\begin{itemize}
    \item Feedback most welcome
    \item Slides and notes are available on the GitHub docs site for the
        course:\\
        {\footnotesize \url{https://paultcochrane.github.io/git\_course/}}
    \item You can submit pull requests, file issues, on GitHub:\\
        {\footnotesize \url{https://github.com/paultcochrane/git\_course/}}
\end{itemize}
\end{frame}

\begin{frame}
\frametitle{About me}
\begin{itemize}
    \item Physicist from New Zealand
    \item Have been involved in scientific computing and development of
        scientific software in Australia and Germany
    \item Led the scientific computing group at the Regional Computing
        Centre for Lower Saxony
    \item Was CTO at a startup in Bremen making polar shipping safer with
        data from space
    \item Active in the Perl language community
    \item Currently freelance software developer with a focus on developing
        and maintaining Python and Perl backend systems.
\end{itemize}
\end{frame}

% vim: expandtab shiftwidth=4 softtabstop=4
